\documentclass[a4paper,11pt]{article}
\usepackage[french]{babel}
\usepackage[T1]{fontenc}
\usepackage[utf8]{inputenc}
\usepackage{lmodern}
\usepackage{microtype}

\usepackage{amsmath,amssymb,bm,upgreek,stmaryrd,mathrsfs,systeme,wasysym}

\usepackage{geometry}
\geometry{hmargin=3cm,vmargin=2cm}

\usepackage{hyperref}

\title{Premier modèle de la Terre comme système énergétique}

\begin{document}
\maketitle

\textbf{Trouver des schémas des situations (ou les faire) !!!} \\

On part des données suivantes :

\begin{enumerate}

\item[•] Puissance surfacique du Soleil reçu par la Terre : $P_S = 1300 ~ W \cdot m^{-2}$
\item[•] Rayon de la Terre : 6371 km
\item[•] Constante de Stefan-Boltzmann : $\sigma = 5,67 \cdot 10^{-8} ~ W \cdot m^{-2} \cdot K^4$ \\

\end{enumerate}

Pour commencer, la puissance que reçoit la Terre correspond au produit de la puissance surfacique et de la surface de la Terre en considérant que la surface de la Terre est un disque pour connaître la puissance réelle reçue sur l'ensemble de sa surface. En notant $P_S$ la puissance surfacique du Soleil à la distance Terre-Soleil du Soleil, S la surface du maître-couple de la Terre et $R_T$ le rayon de la Terre, on a :

\[ P_{reçue} = P_S \cdot S = P_S \cdot \pi R_T^2  \]













\end{document}