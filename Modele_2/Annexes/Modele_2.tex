\documentclass[a4paper,11pt]{article}
\usepackage[frenchb]{babel}
\usepackage[T1]{fontenc}
\usepackage[utf8]{inputenc}
\usepackage{lmodern}
\usepackage{microtype}

\usepackage{array,multirow,makecell}
\setcellgapes{1pt}
\makegapedcells
\usepackage{caption}
\usepackage{adjustbox}

\usepackage{amsmath,amssymb,bm,upgreek,stmaryrd,mathrsfs,systeme}

\usepackage{graphicx}

\usepackage{geometry}
\geometry{hmargin=3cm,vmargin=2cm}

\usepackage{hyperref}

\title{Deuxième modèle de la Terre comme système énergétique}

\begin{document}
\maketitle

\section{Description du modèle}

Dans ce modèle, on propose une amélioration du premier modèle proposé en considérant la position sur Terre (latitude et longitude), ainsi que le temps, c'est-à-dire la partie de la Terre qui est éclairée par le Soleil.

On considère que la Terre suit une trajectoire circulaire. Le rayonnement du soleil arrivant sur la terre est supposé orthogonal (Terre à une distance infinie du Soleil). On ne prend pas encore en compte l'atmosphère. L'albédo est considéré comme une constante A. On prend en compte la latitude et la longitude ainsi que la rotation de la Terre en se basant sur le méridien de Greenwich (heure 0). On néglige l'inclinaison de la Terre.





























\end{document}