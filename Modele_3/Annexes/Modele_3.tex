\documentclass[a4paper,11pt]{article}
\usepackage[frenchb]{babel}
\usepackage[T1]{fontenc}
\usepackage[utf8]{inputenc}
\usepackage{lmodern}
\usepackage{microtype}

\usepackage{array,multirow,makecell}
\setcellgapes{1pt}
\makegapedcells
\usepackage{caption}
\usepackage{adjustbox}

\usepackage{amsmath,amssymb,bm,upgreek,stmaryrd,mathrsfs,systeme}

\usepackage{graphicx}

\usepackage{geometry}
\geometry{hmargin=3cm,vmargin=2cm}

\usepackage{hyperref}

\title{Troisième modèle de la Terre comme système énergétique}

\begin{document}
\maketitle

Dans ce document, nous présentons une troisième version du modèle précédent. Dans un premier temps, nous prenons en compte une différence d'albédo en fonction de l'endroit où l'on se trouve sur Terre au lieu de considérer une moyenne sur l'ensemble de la surface terrestre. Une fois l'albédo en fonction de la position sur Terre prise en compte, nous avons considéré l'atmosphère à travers sa composition, l'effet de serre et celui des aérosols.

Ce document est organisé de la manière suivante : une première section présentant le premier modèle, qui va permettre de comprendre le deuxième modèle dans la section suivante. Ensuite, nous expliquons le troisième modèle tel que décrit précédemment.

Pour tester le modèle, vous pouvez ouvrir le fichier Python "Modele\_3.py" pour faire la simulation.


\section{Modèle 1}

On part des données suivantes :

\begin{enumerate}

\item[•] Puissance surfacique du Soleil reçu par la Terre : $P_S = 1300 ~ W \cdot m^{-2}$
\item[•] Rayon de la Terre : $R_T$ = 6371 km
\item[•] Constante de Stefan-Boltzmann : $\sigma = 5,67 \cdot 10^{-8} ~ W \cdot m^{-2} \cdot K^4$ 
\item[•] Période de rotation de la Terre : T = 1 jour (donc $\omega = \dfrac{2\pi}{T}$ pulsation propre de la Terre)

\end{enumerate}

Pour commencer, la puissance que reçoit la Terre correspond au produit de la puissance surfacique et de la surface de la Terre en considérant que la surface de la Terre est un disque pour connaître la puissance réelle reçue sur l'ensemble de sa surface. En notant $P_S$ la puissance surfacique du Soleil à la distance Terre-Soleil du Soleil, $P_{S/T}$ la puissance reçue par la Terre, S la surface du maître-couple (voir Table 2 dans Annexes) de la Terre et $R_T$ le rayon de la Terre, on a :

\[ P_{S/T} = P_S \cdot S = P_S \cdot \pi R_T^2  \]

Ensuite, en considérant que l'albédo moyen de la Terre est $A = 0.3$, on obtient que la puissance réfléchie $P_r$ par la surface terrestre est :

\[ P_r = A \cdot P_{S/T} \]

En utilisant la loi de Stefan-Boltzmann, on connaît la puissance émise par la Terre $P_e$ :

\[ P_e = \sigma T^4 \cdot 4 \pi R^2 \]

Dans ce modèle, nous négligeons les effets internes de la Terre, ce qui implique que celle-ci n'émet de puissance que par ce qu'elle reçoit du Soleil, soit la puissance absorbée modélisée par la loi de Stefan-Boltzmann. On considère alors que la puissance reçue par la Terre $P_{S/T}$ est égale à la somme de la puissance réfléchie $P_r$ et de la puissance émise $P_e$ :

\[ P_{S/T} = P_r + P_e \] \\

Finalement, en remplaçant par les expressions explicitées précédemment et en isolant la température de la Terre $T$, on obtient :

\[ T = \left(\dfrac{(1 - A) \cdot P_S}{4\sigma}\right)^{1/4} \]

Dans ce modèle, la température est constante et  :

\[ T = -18,57 ^{\circ} C \]

\begin{adjustbox}{center}
\includegraphics[scale=0.9]{Schema_maitre_couple}
\end{adjustbox}
\begin{figure}[h]
  \centering
  \caption{Surface considérée (maître-couple)}
\end{figure}

\section{Modèle 2}

On considère que la Terre suit une trajectoire circulaire. Le rayonnement du soleil arrivant sur la terre est supposé orthogonal (Terre à une distance infinie du Soleil). On ne prend pas encore en compte l’atmosphère. L’albédo est considéré comme une constante A. On prend en compte la latitude et la longitude ainsi que la rotation de la Terre en se basant sur le méridien de Greenwich (heure 0). On néglige l'inclinaison de la Terre. \\

\begin{adjustbox}{center}
\includegraphics[scale=0.32]{Rayons_terre}
\end{adjustbox}
\begin{figure}[h]
  \centering
  \caption{Schéma de la situation}
\end{figure}


Dans ce modèle, la puissance reçue par un élément de surface de la Terre est $P_S \cdot \overrightarrow{dS'}$. De plus, on a $\overrightarrow{dS} = dS \cdot \vec{n}$ et $\overrightarrow{dS'} = dS' \cdot \vec{n'}$. Ainsi, la puissance surfacique reçue par un élément de surface de la Terre est :

\[ P_{S, reçue} = \dfrac{P_S \cdot \overrightarrow{dS} \cdot \overrightarrow{dS'}}{dSdS'} \]

Ce qui donne en simplifiant :

\[ P_{S,reçue} = P_S \cdot \vec{n} \cdot \vec{n'} \]

On projette ensuite $\vec{n}$ et $\vec{n'}$ dans le repère cartésien (x(t), y(t)). Chaque axe dépend de t en raison de la rotation de la Terre sur elle-même, mais ces deux-là sont immobiles dans le référentiel de la Terre.

\begin{adjustbox}{center}
\includegraphics[scale=0.5]{projete_omega_t}
\end{adjustbox}
\begin{figure}[h]
  \centering
  \caption{Rotation de la Terre (vue de dessus)}
\end{figure}

Puisque $ \vec{n} = -\overrightarrow{e_r}$, on a :

\[ \vec{n} = - \sin \theta \cos \phi ~ \vec{x} - \sin \theta \sin \phi ~ \vec{y} \]

En considérant la période de la Terre, on a une projection de $\vec{n'}$ qui va dépendre du temps ($\vec{n'}$ immobile dans le référentiel héliocentrique). Cette projection est la suivante :

\[ \vec{n'} = - \cos \omega t ~ \vec{y} - \sin \omega t ~ \vec{x} \]

Finalement :

\[ P_{S,reçue} = P_S \cdot (- \sin \theta \cos \phi ~ \vec{x} - \sin \theta \sin \phi ~ \vec{y}) \cdot (- \cos \omega t ~ \vec{y} - \sin \omega t ~ \vec{x}) \]

D'où :

\[ P_{S,reçue} = P_S \cdot (\sin \omega t \cdot \sin \theta \cos \phi + \cos \omega t \cdot \sin \theta \sin \phi) \]

Le but étant de rentrer la latitude et la longitude pour déterminer la température en un point de la Terre, on a :

\[ lat = \phi \]
\[ long = \dfrac{\pi}{2} - \theta \leftrightarrow \theta = \dfrac{\pi}{2} - long \]

On peut ensuite calculer la température en un point de la Terre, en fonction de la longitude, la latitude et l'heure de la journée :

\[ T (\theta , \phi , t) = \left(\dfrac{(1 - A) \cdot P_{S,reçue}}{4\sigma}\right)^{1/4} \]

\section{Modèle 3}

Le but de ce modèle était de découper la Terre en plusieurs surfaces en fonction de l'albédo supposée uniforme sur la surface considérée. Pour comprendre comment l'albédo est considérée en fonction de la région du monde, vous pouvez vous référencer au dictionnaire créé sur le code Python associé au modèle 3. Finalement, nous obtenons la découpe suivante :

\begin{adjustbox}{center}
\includegraphics[scale=0.5]{Schema_albedo}
\end{adjustbox}
\begin{figure}[h]
  \centering
  \caption{Répartition de l'albédo dans le modèle}
\end{figure}

Une fois l'albédo considérée de la manière indiquée, nous avons cherché à modéliser l'atmosphère ainsi que l'effet de serre. Nous cherchions à obtenir une relation de récurrence entre la puissance surfacique lors de la n-ème réfraction, et celle lors de la n+1-ème réfraction.

Pour commencer, nous avons considéré un rayon incident, atteignant la surface terrestre, de puissance surfacique égale à 240 $W \cdot m^{-2}$. Ce rayon $u_0$ est supposé entièrement ré-émis par la Terre. Nous avons calculé que 23\% du rayonnement initial ré-émis traversera l'atmosphère sans être arrêté par les gaz à effet de serre (ceci étant dû aux différentes longueurs d'onde : certaines sont moins arrêtées que d'autres par ces gaz).
 
Ensuite, sur la proportion restante (77\% du rayonnement initial), seulement la moitié est renvoyée vers la Terre (50\% de cette énergie est absorbée au niveau des gaz à effet de serre et renvoyée par la Terre).
 
Le même phénomène se reproduit, jusqu'à ce que le rayonnement soit négligeable. Nous pouvons établir la relation de récurrence suivante (un étant la puissance surfacique reçue par la Terre au moment de la n-ième ré-émission vers la Terre) : 

\[ u_{n+1} = 0.5 \cdot 0.77 \cdot u_n ~ , ~ \forall n \geq 1 \]

D'où :

\[ u_n = u_1 * (0.5*0.77)^{n-1} \]

Nous nous intéressons à $ \displaystyle\sum_{k=1}^{n} u_k$. Cette somme tend vers $\dfrac{u_1}{1-0.5 \cdot 0.77} = 244 ~ W \cdot m^{-2}$. En ajoutant les $240 ~W \cdot m^{-2}$ arrivant, on obtient une puissance surfacique totale de $484 ~W \cdot m^{-2}$. 

\newpage

On peut expliquer les calculs grâce au schéma suivant : \\

\begin{adjustbox}{center}
\includegraphics[scale=1]{Effet_de_serre}
\end{adjustbox}
\begin{figure}[h]
  \centering
  \caption{Modélisation de l'effet de serre}
\end{figure}

 










\end{document}